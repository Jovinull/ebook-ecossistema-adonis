\chapter*{Prefácio}
\addcontentsline{toc}{chapter}{Prefácio}

O desenvolvimento web moderno exige muito mais do que apenas saber programar. Ele demanda compreender ecossistemas, ferramentas e boas práticas que, quando bem aplicadas, não só aceleram o desenvolvimento, mas também garantem a escalabilidade, segurança e manutenibilidade de projetos.

O \textbf{AdonisJS} é mais do que um framework. Ele representa uma proposta robusta, opinativa e elegante para o desenvolvimento backend em Node.js, oferecendo uma experiência de desenvolvimento fortemente inspirada nos frameworks maduros do ecossistema PHP e Ruby. Mais do que isso, ele é sustentado por um ecossistema completo: \textbf{Lucid ORM}, \textbf{EdgeJS}, \textbf{VineJS} e \textbf{Japa}, que juntos oferecem soluções integradas para banco de dados, templates, validação e testes.

Este livro surge da necessidade — que também é minha, enquanto desenvolvedor backend que utiliza o AdonisJS profissionalmente — de termos uma fonte única, organizada, objetiva e prática. Um material que sirva tanto como guia de aprendizagem quanto como referência técnica para o dia a dia de quem trabalha com este framework.

Embora a documentação oficial do AdonisJS seja extremamente bem estruturada, a proposta aqui é apresentar o conteúdo de forma sequencial, didática e aprofundada, consolidando os principais conceitos, práticas, padrões de desenvolvimento e, principalmente, trazendo reflexões baseadas na minha experiência real utilizando o AdonisJS em projetos profissionais.

Ao longo destas páginas, você encontrará não apenas comandos, códigos e conceitos, mas também boas práticas, sugestões arquiteturais, alertas sobre armadilhas comuns e dicas valiosas que dificilmente são percebidas na primeira leitura da documentação — e que muitas vezes só são compreendidas no uso prático e na solução de problemas reais.

Este livro foi pensado para quem deseja dominar o \textbf{ecossistema completo do AdonisJS} — desde a criação de APIs robustas, gestão de dados com o \textbf{Lucid ORM}, desenvolvimento de templates dinâmicos com o \textbf{EdgeJS}, até a construção de testes eficientes com o \textbf{Japa} e validações poderosas com o \textbf{VineJS}.

\medskip

\noindent Este material reflete não apenas minha jornada com o AdonisJS, mas também minha trajetória profissional. Atualmente, sou bacharelando em Sistemas de Informação pelo Instituto Federal de Sergipe (IFS), com sólida atuação em desenvolvimento Full Stack utilizando tecnologias como \textbf{AdonisJS}, \textbf{Next.js}, \textbf{TypeScript} e \textbf{PostgreSQL}, além de uma ampla vivência em \textbf{microeletrônica} e \textbf{sistemas embarcados} com \textbf{ESP32}.

Atuei como pesquisador no \textbf{Laboratório de Inovação e Criatividade (LABIC)}, desenvolvendo projetos que unem \textbf{Inteligência Artificial}, eletrônica e sistemas embarcados. Fui participante da \textbf{Mostra Nacional de Robótica 2023}, além de autor de publicações acadêmicas sobre acesso à formação em IA para pessoas de baixa renda. Minha atuação também incluiu papel como \textbf{instrutor} e \textbf{mentor} em diversos projetos de capacitação tecnológica, como o \textbf{Projeto Aprendiz 4.0} e a \textbf{Residência em TIC}.

Hoje também compartilho conhecimento como \textbf{criador de conteúdo educacional} no YouTube, buscando democratizar o acesso à tecnologia e capacitar profissionais da área de desenvolvimento de software.

Minha missão com este livro é a mesma que carrego em toda minha trajetória: \textbf{tornar o conhecimento acessível}, descomplicado e aplicável. Que esta obra te ajude não só a dominar o AdonisJS, mas também a construir projetos mais profissionais, escaláveis e de alta qualidade.

\medskip

\noindent Boa leitura e bons códigos!

\bigskip

\noindent \textbf{Felipe Jovino dos Santos} \\
\textit{Desenvolvedor Full Stack | Especialista em AdonisJS | Criador de Conteúdo | Pesquisador em IA e Sistemas Embarcados}
