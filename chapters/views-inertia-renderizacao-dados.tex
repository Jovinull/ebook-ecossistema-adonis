\section{Renderização, root template, compartilhamento de dados e CSRF}

\subsection{Renderizando páginas e recebendo props}

O método \verb|inertia.render| recebe:

\begin{enumerate}
  \item Nome do componente (relativo a \verb|inertia/pages|).
  \item Um objeto de dados que será serializado em JSON e enviado como props.
  \item Opcionalmente, um objeto extra com dados para a template root.
\end{enumerate}

Exemplo simples:

\lstinputlisting[
  language=TypeScript,
  caption={Controller usando inertia.render},
  label={lst:inertia_render}
]{snippets/templates/inertia_render.ts}

No componente Vue:

\lstinputlisting[
  language=html,
  caption={Componente recebendo props},
  label={lst:inertia_component_props}
]{snippets/templates/inertia_component_props.vue}

\begin{warningbox}
  Tudo que você passa para \verb|inertia.render| é serializado em JSON.
  Não espere receber instâncias de \texttt{Model}, \texttt{Date} ou objetos complexos intactos no frontend.
\end{warningbox}

\subsection{Template root (Edge) para Inertia}

A primeira visita da aplicação é servida como uma página HTML normal, usando uma \textbf{template root} Edge. É ali que você:

\begin{itemize}
  \item Define o \texttt{<head>} e as meta tags.
  \item Inclui CSS/JS via \verb|@vite|.
  \item Coloca as diretivas \verb|@inertiaHead()| e \verb|@inertia()|.
\end{itemize}

Exemplo típico:

\lstinputlisting[
  language=html,
  caption={Template root com Inertia},
  label={lst:inertia_root_template}
]{snippets/templates/inertia_root_template.edge}

Por padrão, o caminho desse template é configurado em \verb|config/inertia.ts| como \verb|resources/views/inertia_layout.edge|, mas você pode alterar:

\lstinputlisting[
  language=TypeScript,
  caption={Configurando a root view},
  label={lst:inertia_root_config}
]{snippets/templates/inertia_root_config.ts}

Também é possível decidir dinamicamente qual layout usar:

\lstinputlisting[
  language=TypeScript,
  caption={rootView dinâmico},
  label={lst:inertia_root_dynamic}
]{snippets/templates/inertia_root_dynamic.ts}

\subsection{Dados para a template root}

Além das props enviadas para o componente Inertia, você pode passar dados extras para a template root no \textbf{terceiro argumento} de \verb|inertia.render|:

\lstinputlisting[
  language=TypeScript,
  caption={Metadados para a root view},
  label={lst:inertia_root_data}
]{snippets/templates/inertia_root_data.ts}

Na template root, esses dados são acessíveis como variáveis Edge:

\lstinputlisting[
  language=html,
  caption={Usando dados extras na root view},
  label={lst:inertia_root_data_use}
]{snippets/templates/inertia_root_data_use.edge}

\subsection{Compartilhando dados com todas as páginas}

Em vez de repetir sempre a mesma informação em todos os controllers, podemos usar:

\begin{itemize}
  \item \textbf{\texttt{sharedData}} em \verb|config/inertia.ts|.
  \item \textbf{\texttt{inertia.share}} em um middleware.
\end{itemize}

\subsubsection{sharedData na configuração}

\lstinputlisting[
  language=TypeScript,
  caption={Dados globais via sharedData},
  label={lst:inertia_shared_config}
]{snippets/templates/inertia_shared_config.ts}

\subsubsection{Compartilhando via middleware}

\lstinputlisting[
  language=TypeScript,
  caption={Compartilhando dados em middleware},
  label={lst:inertia_shared_middleware}
]{snippets/templates/inertia_shared_middleware.ts}

\subsection{Redirects com Inertia}

Redirecionar com Inertia segue o mesmo padrão do AdonisJS:

\lstinputlisting[
  language=TypeScript,
  caption={Redirect interno padrão},
  label={lst:inertia_redirect_internal}
]{snippets/templates/inertia_redirect_internal.ts}

Para redirecionar para URLs externas, use \verb|inertia.location|:

\lstinputlisting[
  language=TypeScript,
  caption={Redirect externo com inertia.location},
  label={lst:inertia_redirect_external}
]{snippets/templates/inertia_redirect_external.ts}

\subsection{Partial reloads e lazy data}

O Inertia permite \textbf{recarregar apenas parte} das props de uma página. Para isso, você pode:

\begin{itemize}
  \item Passar funções em vez de valores (lazy).
  \item Usar \verb|inertia.optional| para dados que não vão na primeira visita.
\end{itemize}

Exemplo:

\lstinputlisting[
  language=TypeScript,
  caption={Lazy evaluation e optional data},
  label={lst:inertia_lazy}
]{snippets/templates/inertia_lazy.ts}

\subsection{CSRF e Inertia}

Se a sua aplicação usa proteção CSRF (\texttt{@adonisjs/shield}), basta habilitar a opção \verb|enableXsrfCookie| em \verb|config/shield.ts|:

\lstinputlisting[
  language=TypeScript,
  caption={Habilitando cookie XSRF},
  label={lst:shield_xsrf}
]{snippets/templates/shield_xsrf.ts}

Isso faz com que o cookie \texttt{XSRF-TOKEN} seja setado no cliente e reenviado em cada requisição. Não é necessário ajuste extra específico para o Inertia: ele se beneficia da mesma proteção que o restante da aplicação.
