\section{EdgeJS: instalação, primeiros templates e passagem de dados}

\subsection{Instalando e registrando o EdgeJS}

Edge é o template engine oficial do AdonisJS. A instalação é feita via CLI:

\lstinputlisting[
  language=bash,
  caption={Instalando o EdgeJS},
  label={lst:edge_install}
]{snippets/templates/edge_install.sh}

O comando:

\begin{itemize}
  \item Adiciona as dependências necessárias.
  \item Cria a configuração padrão.
  \item Registra o serviço de views para que você possa usar \verb|ctx.view|.
\end{itemize}

\subsection{Criando o primeiro template}

Por convenção, os templates ficam em \verb|resources/views|. Vamos criar uma view chamada \texttt{welcome}:

\lstinputlisting[
  language=bash,
  caption={Gerando uma view com a CLI},
  label={lst:edge_make_view}
]{snippets/templates/edge_make_view.sh}

Esse comando cria o arquivo:

\begin{center}
  \verb|resources/views/welcome.edge|
\end{center}

Exemplo de conteúdo inicial:

\lstinputlisting[
  language=html,
  caption={Template básico Edge},
  label={lst:edge_welcome}
]{snippets/templates/edge_welcome.edge}

Repare que:

\begin{itemize}
  \item O HTML é padrão.
  \item As chaves \verb|{{ ... }}| interpolam valores (similar a outras engines).
  \item Temos acesso a helpers como \verb|request.url()| direto na template.
\end{itemize}

\subsection{Renderizando views via rotas}

Para servir a view, usamos o serviço de \texttt{router} e o método \texttt{view.render} no contexto HTTP:

\lstinputlisting[
  language=TypeScript,
  caption={Renderizando uma view no controller/rota},
  label={lst:edge_render_route}
]{snippets/templates/edge_render_route.ts}

No exemplo acima:

\begin{itemize}
  \item O nome \texttt{'welcome'} corresponde ao arquivo \texttt{welcome.edge}.
  \item Não é necessário passar a extensão nem o caminho completo (Adonis já sabe a pasta padrão).
\end{itemize}

Existe também um atalho quando você não precisa de um callback:

\lstinputlisting[
  language=TypeScript,
  caption={Atalho com \texttt{router.on().render()}},
  label={lst:edge_router_on}
]{snippets/templates/edge_router_on.ts}

\subsection{Passando dados para a view}

Para enviar dados dinâmicos, passamos um objeto como segundo argumento de \texttt{view.render}:

\lstinputlisting[
  language=TypeScript,
  caption={Passando contexto para o template},
  label={lst:edge_pass_data}
]{snippets/templates/edge_pass_data.ts}

No template, podemos usar essa variável diretamente:

\lstinputlisting[
  language=html,
  caption={Consumindo dados no Edge},
  label={lst:edge_use_data}
]{snippets/templates/edge_use_data.edge}

\begin{tipbox}
  Uma prática comum é sempre passar \textbf{objetos estruturados} (por exemplo, \verb|{ user, stats, filters }|) em vez de muitas variáveis soltas. Isso facilita a evolução do layout sem quebrar templates antigos.
\end{tipbox}

\subsection{Views, layout base e Vite}

Embora a documentação desta seção foque no Edge puro, em projetos reais você normalmente:

\begin{itemize}
  \item Define um \textbf{layout base} (ex.: \verb|layouts/base.edge|) que inclui \verb|@vite| para carregar CSS/JS.
  \item Usa diretivas do Edge para definir seções, herança, includes e componentes.
\end{itemize}

Exemplo simplificado de layout base:

\lstinputlisting[
  language=html,
  caption={Layout base Edge com Vite},
  label={lst:edge_base_layout}
]{snippets/templates/edge_base_layout.edge}

E uma view que estende esse layout:

\lstinputlisting[
  language=html,
  caption={View que estende o layout},
  label={lst:edge_extend_layout}
]{snippets/templates/edge_extend_layout.edge}
