\chapter{Recursos Avançados}

\section{Cache com \texttt{@adonisjs/cache}}

\subsection{Visão geral}
O \texttt{@adonisjs/cache} é um wrapper leve sobre o BentoCache, oferecendo uma API unificada para cachear dados e melhorar desempenho. Ele suporta múltiplos drivers (Redis, DynamoDB, PostgreSQL, in-memory etc.) e recursos avançados como \textit{multi-tier}, \textit{grace period}, \textit{timeouts}, \textit{tagging} e \textit{namespaces}.

\begin{tipbox}
    Para uso avançado (multi-tier, proteção contra \textit{stampede}, tags e \textit{grace}), vale ler a documentação do BentoCache, pois o AdonisJS expõe esses conceitos de forma direta.
\end{tipbox}

\subsection{Instalação}
\lstinputlisting[
    language=bash,
    caption={Instalando o pacote de cache},
    label={lst:cache_install}
]{snippets/cache/install_cache.sh}

\subsection{Configuração}
O arquivo de configuração fica em \texttt{config/cache.ts}. Você define o \texttt{default} e os \texttt{stores}.

\lstinputlisting[
    language=TypeScript,
    caption={Exemplo de configuração do cache com múltiplas camadas},
    label={lst:cache_config}
]{snippets/cache/cache_config.ts}

\subsection{Redis e Database}
Para usar Redis, você precisa do \texttt{@adonisjs/redis} configurado. Para usar o driver de banco, você precisa do \texttt{@adonisjs/lucid} e executar a migration publicada para armazenar as entradas de cache.

\subsection{Uso}
\subsection{Cacheando via \texttt{getOrSet}}
\lstinputlisting[
    language=TypeScript,
    caption={Exemplo de uso do cache para armazenar dados por TTL},
    label={lst:cache_usage_route}
]{snippets/cache/cache_usage_route.ts}

\begin{infobox}
    Caches distribuídos exigem serialização: modelos Lucid, \texttt{Date} e instâncias de classe devem ser convertidos (ex.: \texttt{toJSON()}).
\end{infobox}

\subsection{Tagging e Namespaces}
\subsection{Tagging}
\lstinputlisting[
    language=TypeScript,
    caption={Cache com tags e invalidação por tag},
    label={lst:cache_tagging}
]{snippets/cache/cache_tagging.ts}

\subsection{Namespaces}
\lstinputlisting[
    language=TypeScript,
    caption={Agrupando chaves com namespaces},
    label={lst:cache_namespaces}
]{snippets/cache/cache_namespaces.ts}

\subsection{Grace period e Timeouts}
\subsection{Grace period}
\lstinputlisting[
    language=TypeScript,
    caption={Grace period para retornar dado stale enquanto revalida},
    label={lst:cache_grace_timeout}
]{snippets/cache/cache_grace_timeout.ts}

\subsection{Hard timeout}
\lstinputlisting[
    language=TypeScript,
    caption={Hard timeout para interromper a factory após limite},
    label={lst:cache_hard_timeout}
]{snippets/cache/cache_hard_timeout.ts}

\subsection{Cache service e Edge helper}
\subsection{Alternando stores com \texttt{use}}
\lstinputlisting[
    language=TypeScript,
    caption={Alternando o store em tempo de execução},
    label={lst:cache_service_use}
]{snippets/cache/cache_service_use.ts}

\subsection{Usando no Edge}
\lstinputlisting[
    language=html,
    caption={Lendo cache diretamente no Edge (async helper)},
    label={lst:cache_edge_helper}
]{snippets/cache/cache_edge_helper.edge}

\subsection{Ace commands}
\lstinputlisting[
    language=bash,
    caption={Limpando cache (default e stores específicos)},
    label={lst:cache_ace_clear}
]{snippets/cache/cache_ace_clear.sh}

\lstinputlisting[
    language=bash,
    caption={Removendo uma chave de cache},
    label={lst:cache_ace_delete}
]{snippets/cache/cache_ace_delete.sh}

\lstinputlisting[
    language=bash,
    caption={Podando chaves expiradas (drivers sem TTL nativo)},
    label={lst:cache_ace_prune}
]{snippets/cache/cache_ace_prune.sh}
