\chapter{Primeiros Passos}
\section{Instalação}

\begin{infobox}
  Antes de começar, é fundamental ter instalado em seu ambiente o \textbf{Node.js} na versão \textbf{20 ou superior}. O AdonisJS depende diretamente dessa versão mínima para funcionar corretamente.
\end{infobox}

\subsection{Verificando o Node.js}

Verifique se possui o Node.js instalado com o seguinte comando no terminal:

\lstinputlisting[
  language=bash,
  caption={Verificando a versão do Node.js},
  label={lst:node_version_check}
]{snippets/primeiros-passos/node_version_check.sh}

Se não tiver, você pode instalar diretamente pelo site oficial \url{https://nodejs.org} ou, preferencialmente, utilizar o \textbf{Volta} (\url{https://volta.sh}) — um gerenciador de versões que permite maior controle e consistência no ambiente de desenvolvimento.

\subsection{Criando um Novo Projeto}

O AdonisJS oferece um comando simples para inicializar um novo projeto utilizando o seguinte comando:

\lstinputlisting[
  language=bash,
  caption={Criando um novo projeto AdonisJS},
  label={lst:adonis_new_project}
]{snippets/primeiros-passos/adonis_new_project.sh}

\begin{warningbox}
  Quando utilizar parâmetros no comando \texttt{npm init}, lembre-se de adicionar \texttt{--} (dois hífens) antes dos parâmetros para garantir que o npm os repasse corretamente.
\end{warningbox}

\subsection{Parâmetros mais comuns}

Você pode personalizar seu projeto logo na criação usando os seguintes parâmetros:

\begin{itemize}
  \item \texttt{--kit}: Define o kit inicial do projeto. Opções: \texttt{web}, \texttt{api}, \texttt{slim}, \texttt{inertia}.
  \item \texttt{--db}: Define o banco de dados. Opções: \texttt{sqlite}, \texttt{postgres}, \texttt{mysql}, \texttt{mssql}.
  \item \texttt{--git-init}: Inicializa um repositório Git (opcional).
  \item \texttt{--auth-guard}: Define o tipo de autenticação: \texttt{session}, \texttt{access\_tokens} ou \texttt{basic\_auth}.
\end{itemize}

\subsection{Exemplos práticos}

\begin{examplebox}
  Criar um projeto básico, perguntando todas as opções:
\end{examplebox}

\lstinputlisting[
  language=bash,
  caption={Criando um novo projeto AdonisJS},
  label={lst:adonis_new_project_basic}
]{snippets/primeiros-passos/adonis_new_project_basic.sh}

\begin{examplebox}
  Criar um projeto API usando PostgreSQL:
\end{examplebox}

\lstinputlisting[
  language=bash,
  caption={Projeto API com PostgreSQL},
  label={lst:adonis_new_project_api_postgres}
]{snippets/primeiros-passos/adonis_new_project_api_postgres.sh}

\begin{examplebox}
  Criar um projeto Web com MySQL:
\end{examplebox}

\lstinputlisting[
  language=bash,
  caption={Projeto Web com MySQL},
  label={lst:adonis_new_project_web_mysql}
]{snippets/primeiros-passos/adonis_new_project_web_mysql.sh}

\begin{examplebox}
  Criar um projeto com autenticação baseada em tokens:
\end{examplebox}

\lstinputlisting[
  language=bash,
  caption={Projeto API com guard de autenticação por tokens},
  label={lst:adonis_new_project_api_tokens}
]{snippets/primeiros-passos/adonis_new_project_api_tokens.sh}

\subsection{Escolhendo o Starter Kit}

O AdonisJS oferece quatro tipos principais de starter kits, cada um pensado para um tipo de projeto:

\begin{description}
  \item[\textbf{Web}] Para aplicações que renderizam HTML no backend usando EdgeJS.
  \item[\textbf{API}] Ideal para backends que servem dados em JSON (para apps mobile, frontend em React, Vue, etc.).
  \item[\textbf{Slim}] Uma versão mínima do framework, sem ORM, sem autenticação e sem templates.
  \item[\textbf{Inertia}] Para quem deseja construir SPAs server-driven, usando React, Vue, Solid ou Svelte no frontend.
\end{description}

\begin{tipbox}
  Se sua aplicação não precisa de um frontend robusto, o kit \texttt{web} oferece extrema produtividade utilizando apenas EdgeJS para renderização no servidor.
\end{tipbox}

\subsection{Iniciando o servidor de desenvolvimento}

Após criar o projeto, execute o servidor com:

\lstinputlisting[
  language=bash,
  caption={Iniciando o servidor},
  label={lst:server_start}
]{snippets/primeiros-passos/server_start.sh}

Acesse no navegador:

\lstinputlisting[
  language=bash,
  caption={URL local da aplicação},
  label={lst:open_localhost}
]{snippets/primeiros-passos/open_localhost.txt}

\subsection{Build para produção}

Compile sua aplicação para produção com:

\lstinputlisting[
  language=bash,
  caption={Gerando build de produção},
  label={lst:build_production}
]{snippets/primeiros-passos/build_production.sh}

O código será gerado na pasta \texttt{build/} e está pronto para ser executado com Node.js.

\subsection{Configurando o ambiente de desenvolvimento}

O projeto já vem com \texttt{ESLint} e \texttt{Prettier} configurados com os padrões recomendados pela equipe do AdonisJS.

\lstinputlisting[
  language=bash,
  caption={Executando ferramentas de lint e format},
  label={lst:lint_format}
]{snippets/primeiros-passos/lint_format.sh}

\begin{tipbox}
  Recomendo instalar as extensões do AdonisJS, EdgeJS e Japa para VSCode. Elas oferecem recursos como:
  \begin{itemize}
    \item Highlight e snippets.
    \item Rodar comandos Ace direto do VSCode.
    \item Visualizar rotas e migrations.
  \end{itemize}
\end{tipbox}

\section{Primeiro Projeto}

Ao criar um projeto com AdonisJS, você não está apenas iniciando um backend. Você está construindo uma aplicação com uma arquitetura sólida, uma base fortemente tipada e uma série de ferramentas integradas que tornam o desenvolvimento muito mais produtivo.

Depois de executar:

\lstinputlisting[
  language=bash,
  caption={Criando o projeto},
  label={lst:adonis_new_project_basic_others}
]{snippets/primeiros-passos/adonis_new_project_basic.sh}

E rodar:

\lstinputlisting[
  language=bash,
  caption={Iniciando o servidor},
  label={lst:server_starter}
]{snippets/primeiros-passos/server_start.sh}

Você verá sua aplicação rodando em:

\lstinputlisting[
  language=bash,
  caption={URL local da aplicação},
  label={lst:open_the_localhost}
]{snippets/primeiros-passos/open_localhost.txt}

Por padrão, o projeto já vem com:

\begin{itemize}
  \item Uma API pronta (se você escolheu o kit \texttt{api}).
  \item Estrutura MVC bem definida.
  \item Gerenciamento de banco com Lucid ORM.
  \item Sistema de autenticação (se escolhido).
  \item Validações robustas com VineJS.
  \item Logs, tratamento de erros, middlewares e muito mais.
\end{itemize}

\begin{infobox}
  O projeto já está totalmente funcional após o primeiro comando. O AdonisJS gera uma estrutura produtiva desde o primeiro momento, permitindo que você foque no que realmente importa: as regras de negócio.
\end{infobox}

\section{Estrutura de Diretórios}

\begin{tipbox}
  O AdonisJS oferece uma estrutura de diretórios opinativa, pensada para escalar e manter sua aplicação organizada. Entretanto, você tem total liberdade para ajustá-la conforme as necessidades do seu projeto.
\end{tipbox}

A seguir, uma visão geral da estrutura padrão:

\lstinputlisting[
  language=bash,
  caption={Estrutura de Diretórios},
  label={lst:project_structure}
]{snippets/primeiros-passos/project_structure.txt}

\subsection*{Principais Diretórios}

\begin{description}
  \item[\textbf{app/}] Onde vive sua lógica de negócio. Controllers, Models, Services, Validators, Middlewares, Listeners e Events.
  \item[\textbf{config/}] Arquivos de configuração (banco, cache, autenticação, etc.).
  \item[\textbf{start/}] Arquivos carregados na inicialização da aplicação (\texttt{routes.ts}, \texttt{kernel.ts}, \texttt{events.ts}, \texttt{env.ts}).
  \item[\textbf{resources/}] Views EdgeJS e assets do frontend.
  \item[\textbf{database/}] Migrations e Seeders.
  \item[\textbf{providers/}] Service Providers para extender funcionalidades do framework.
  \item[\textbf{types/}] Arquivos TypeScript para contratos e definições de tipos.
  \item[\textbf{public/}] Arquivos estáticos acessíveis externamente (\texttt{/style.css} → \url{http://localhost:3333/style.css}).
\end{description}

\begin{examplebox}
  O arquivo \texttt{start/routes.ts} é onde você define suas rotas HTTP.
  O \texttt{app/controllers/} guarda os controllers que recebem essas requisições.
  O \texttt{app/services/} pode ser usado para regras de negócio mais complexas.
\end{examplebox}

\section{Ambiente e Configurações}

A configuração no AdonisJS é simples, escalável e baseada em dois pilares fundamentais:

\begin{itemize}
  \item Arquivos dentro da pasta \texttt{config/}.
  \item Variáveis de ambiente (\texttt{.env}).
\end{itemize}

\subsection{Configurações do Projeto}

Todos os arquivos da pasta \texttt{config/} são módulos TypeScript e podem ser importados diretamente:

\lstinputlisting[
  language=TypeScript,
  caption={Importando diretamente},
  label={lst:config_import_direct}
]{snippets/primeiros-passos/import_config_direct.ts}

Ou, de forma dinâmica, utilizando o \texttt{config service}:

\lstinputlisting[
  language=TypeScript,
  caption={Acessando via Config Service},
  label={lst:config_service_access}
]{snippets/primeiros-passos/config_service_access.ts}

\begin{tipbox}
  Usar o \texttt{config service} permite acessar as configurações dentro de Providers, Edge Templates e ambientes onde importações diretas não funcionam.
\end{tipbox}

\subsection{Variáveis de Ambiente}

O arquivo \texttt{start/env.ts} controla as variáveis de ambiente, valida suas existências e tipos:

\lstinputlisting[
  language=TypeScript,
  caption={Exemplo básico de env.ts},
  label={lst:env_ts_example}
]{snippets/primeiros-passos/env_ts_example.ts}

O \texttt{env.get()} é utilizado para acessar os valores:

\lstinputlisting[
  language=TypeScript,
  caption={Lendo valores do .env},
  label={lst:env_read_values}
]{snippets/primeiros-passos/env_read_values.ts}

\begin{warningbox}
  Se uma variável não estiver declarada no arquivo \texttt{start/env.ts}, o AdonisJS acusará erro de boot — isso garante segurança e previsibilidade.
\end{warningbox}

\subsection{Estrutura do .env}

\lstinputlisting[
  language=bash,
  caption={Exemplo de .env},
  label={lst:dotenv_example}
]{snippets/primeiros-passos/dotenv_example.env}

\subsection{Configurações no Edge Templates}

\lstinputlisting[
  language=html,
  caption={Utilizando config e env nas views},
  label={lst:edge_config_env}
]{snippets/primeiros-passos/edge_config_env.edge}

\begin{infobox}
  O acesso a variáveis no Edge é restrito ao server-side. Nenhuma informação sensível vaza para o cliente.
\end{infobox}

\subsection{Alterando Configuração em Tempo de Execução}

Você pode alterar valores durante a execução:

\lstinputlisting[
  language=TypeScript,
  caption={Alterando valor de configuração em tempo de execução},
  label={lst:config_runtime_set}
]{snippets/primeiros-passos/config_runtime_set.ts}

\begin{warningbox}
  Essa alteração é feita apenas em memória, válida para toda a instância da aplicação até ser reiniciada. Isso não altera os arquivos no disco.
\end{warningbox}

\begin{tipbox}
  Boas práticas:
  \begin{itemize}
    \item Nunca versionar arquivos \texttt{.env} em repositórios públicos.
    \item Utilizar chaves seguras para \texttt{APP\_KEY} e senhas de banco.
    \item Configurar múltiplos arquivos \texttt{.env} para ambientes diferentes: \texttt{.env.development.local}, \texttt{.env.production}.
  \end{itemize}
\end{tipbox}
