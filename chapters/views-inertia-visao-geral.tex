\section{InertiaJS: visão geral, instalação e entrypoint do frontend}

\subsection{O que é o InertiaJS no contexto do AdonisJS}

O InertiaJS é uma camada que transforma seu backend AdonisJS em “backend para SPA” sem obrigar você a criar uma API REST separada.

Em vez de:

\begin{itemize}
  \item Definir rotas que retornam JSON.
  \item Consumir essa API via fetch/Axios no frontend.
\end{itemize}

Você define controllers AdonisJS que chamam \verb|inertia.render|, e o Inertia cuida de:

\begin{itemize}
  \item Trocar dados entre backend e frontend.
  \item Atualizar o histórico de navegação.
  \item Renderizar o componente certo no browser.
\end{itemize}

\subsection{Instalação e configuração}

Para iniciar um projeto novo já com Inertia, o recomendado é usar o \textbf{Inertia starter kit}. Mas, se você quiser adicionar Inertia a um projeto existente:

\lstinputlisting[
  language=bash,
  caption={Instalando Inertia no AdonisJS},
  label={lst:inertia_install}
]{snippets/templates/inertia_install.sh}

O comando de configuração:

\begin{itemize}
  \item Registra o middleware \texttt{inertia\_middleware} em \verb|start/kernel.ts|.
  \item Cria a estrutura \verb|inertia/app| e \verb|inertia/pages|.
  \item Cria um template root Edge para usar Inertia.
\end{itemize}

Depois disso, basta subir o servidor de desenvolvimento (\verb|node ace serve --watch|) e acessar \verb|http://localhost:3333| para ver a página inicial renderizada via Inertia + o framework frontend escolhido.

\subsection{Entrypoint do frontend (cliente)}

O Inertia precisa de um \textbf{entrypoint} do lado do cliente, responsável por:

\begin{itemize}
  \item Inicializar o framework (Vue, React, etc.).
  \item Resolver o componente da página com base no nome passado pelo backend.
  \item Montar a aplicação na árvore DOM.
\end{itemize}

Exemplo de entrypoint para Vue 3:

\lstinputlisting[
  language=TypeScript,
  caption={\texttt{inertia/app/app.ts} (cliente Vue 3)},
  label={lst:inertia_client_entry}
]{snippets/templates/inertia_client_entry.ts}

Pontos importantes:

\begin{itemize}
  \item \textbf{\texttt{resolve}}: recebe o nome da página (ex.: \texttt{'users/index'}) e encontra o componente correspondente em \verb|inertia/pages|.
  \item \textbf{\texttt{setup}}: monta a aplicação Vue (ou React/Svelte/Solid) no elemento \verb|el| fornecido pelo Inertia.
  \item \textbf{\texttt{@vite}}: esse arquivo é carregado pela template root através da diretiva \verb|@vite|.
\end{itemize}

\begin{infobox}
  No fluxo Inertia, você não pensa em “rotas React” ou “rotas Vue”. As rotas continuam sendo AdonisJS; os componentes são apenas “views ricas” que recebem dados via \verb|inertia.render|.
\end{infobox}

\subsection{Middleware do Inertia}

O middleware registrado em \verb|start/kernel.ts| garante que toda requisição tenha acesso a \verb|ctx.inertia|:

\lstinputlisting[
  language=TypeScript,
  caption={Uso de \texttt{inertia.render} em um controller},
  label={lst:inertia_controller_basic}
]{snippets/templates/inertia_controller_basic.ts}

O primeiro parâmetro (\texttt{'users/index'}) é mapeado para o arquivo \verb|inertia/pages/users/index.vue| (ou \verb|.tsx|, etc.), e o segundo são as \textbf{props} enviadas para o componente.
