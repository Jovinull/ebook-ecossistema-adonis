\section{Tabela de usuários e ajustes no modelo}

Ao configurar o Auth, o AdonisJS gera uma migration para a tabela \texttt{users}. Ela fica em \verb|database/migrations| e é o ponto onde você ajusta a estrutura para o seu domínio.

Por padrão, a migration cria colunas como:

\lstinputlisting[
    language=TypeScript,
    caption={Migration padrão da tabela \texttt{users}},
    label={lst:auth_users_migration_default}
]{snippets/auth/auth_users_migration_default.ts}

Se você renomear ou remover colunas, lembre-se de atualizar o \textbf{User model} também.

\begin{tipbox}
    Em aplicações reais, eu quase sempre adiciono campos como \verb|username|, \verb|phone| ou \verb|is_active|. O ponto é: defina cedo quais campos serão \textbf{UIDs} de login (email? username? telefone?) para não “remendar” depois.
\end{tipbox}

\section{Próximos passos do Auth}

Depois da instalação, o fluxo natural é:

\begin{enumerate}
    \item Verificar credenciais de forma segura (evitar timing attacks).
    \item Implementar autenticação stateful com \textbf{session guard}.
    \item Implementar autenticação stateless com \textbf{access tokens}.
\end{enumerate}
