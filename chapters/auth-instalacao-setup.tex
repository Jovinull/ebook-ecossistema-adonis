\section{Instalação e configuração inicial}

O sistema de autenticação já vem pré-configurado nos starter kits \textbf{web} e \textbf{api}. Porém, se você está montando o projeto manualmente, pode instalar via CLI escolhendo o guard principal.

\lstinputlisting[
    language=bash,
    caption={Instalando o pacote de Auth com guards diferentes},
    label={lst:auth_install}
]{snippets/auth/auth_install.sh}

\subsection{O middleware de inicialização do Auth}

Durante o setup, o AdonisJS registra o middleware \textbf{InitializeAuthMiddleware}. Esse middleware:

\begin{itemize}
    \item instancia a classe \textbf{Authenticator};
    \item expõe a instância em \verb|ctx.auth| para o resto do request.
\end{itemize}

Um detalhe importante: \textbf{ele não autentica a requisição}. Ele apenas inicializa o Authenticator.

\begin{warningbox}
    Para proteger rotas, você deve usar o middleware \textbf{auth}. Iniciar Auth \(\neq\) autenticar.
\end{warningbox}

\subsection{Auth disponível em templates Edge}

Se seu projeto usa EdgeJS, a mesma instância de \verb|ctx.auth| é compartilhada com templates como \verb|auth|. Por exemplo:

\lstinputlisting[
    language=html,
    caption={Acessando \texttt{auth} dentro de templates Edge},
    label={lst:auth_edge_usage}
]{snippets/auth/auth_edge_usage.edge}
