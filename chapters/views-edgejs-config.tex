\section{EdgeJS avançado: preload, plugins e helpers globais}

Depois de ter o básico funcionando, é comum querer:

\begin{itemize}
  \item Adicionar \textbf{plugins} ao Edge.
  \item Criar \textbf{variáveis globais} visíveis em todas as views.
  \item Centralizar helpers de layout, ícones, URLs, etc.
\end{itemize}

O local recomendado para isso é um \textbf{arquivo de preload}.

\subsection{Criando o preload de views}

Use a CLI para gerar um preload relacionado a views:

\lstinputlisting[
  language=bash,
  caption={Gerando preload para Edge},
  label={lst:edge_make_preload}
]{snippets/templates/edge_make_preload.sh}

Esse comando cria algo como \verb|start/view.ts|. É nele que vamos estender o comportamento padrão do Edge.

\subsection{Registrando plugins}

Plugins são extensões que adicionam novas tags, filtros ou funcionalidades à engine. Exemplo com \texttt{edge-iconify}:

\lstinputlisting[
  language=TypeScript,
  caption={Registrando plugin no Edge},
  label={lst:edge_plugin}
]{snippets/templates/edge_plugin.ts}

A partir daí, todas as views podem usar os recursos desse plugin sem nenhuma configuração adicional.

\subsection{Globais: compartilhando dados e helpers com todas as views}

O método \texttt{edge.global} permite definir variáveis disponíveis em todas as templates:

\lstinputlisting[
  language=TypeScript,
  caption={Definindo globais no preload},
  label={lst:edge_global}
]{snippets/templates/edge_global.ts}

No Edge:

\lstinputlisting[
  language=html,
  caption={Usando globais nas views},
  label={lst:edge_global_use}
]{snippets/templates/edge_global_use.edge}

\begin{infobox}
  Globais são ideais para itens realmente “de aplicação”: \texttt{appUrl}, nome do sistema, dados de branding, helpers de formatação, etc. Para dados de usuário/logado, prefira compartilhar via contexto individual (por controller, middleware ou Inertia).
\end{infobox}

\subsection{Helpers do Edge e helpers do AdonisJS}

Além dos globais definidos por você, o Edge e o AdonisJS expõem uma série de \textbf{helpers} padrão:

\begin{itemize}
  \item Helpers de controle de fluxo: \verb|@if|, \verb|@each|, \verb|@hasSection|, etc.
  \item Helpers integrados ao AdonisJS (como o \texttt{csrfField()}, usado na proteção CSRF).
\end{itemize}

A documentação oficial lista todos os helpers disponíveis. Ao longo do livro, eu uso principalmente:

\begin{itemize}
  \item \verb|csrfField()| em formulários.
  \item Helpers de sessão/flash para exibir mensagens de sucesso/erro.
  \item Diretivas de \textbf{autorização} quando Bouncer está ligado às views.
\end{itemize}

\subsection{Edge + htmx/unpoly: enriquecendo sem virar SPA}

Uma estratégia muito usada em projetos “clássicos” é combinar:

\begin{itemize}
  \item Edge como motor principal de views.
  \item htmx ou Unpoly para navegação parcial (fetch de fragmentos HTML).
\end{itemize}

Do ponto de vista de AdonisJS:

\begin{itemize}
  \item As rotas continuam renderizando templates Edge.
  \item htmx/unpoly controlam quais trechos são substituídos no DOM.
\end{itemize}

Isso te dá interatividade extra sem migrar para o modelo de SPA + Inertia. Em sistemas administrativos tradicionais é um meio-termo bem confortável.
